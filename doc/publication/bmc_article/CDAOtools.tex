%% BioMed_Central_Tex_Template_v1.06
%%                                      %
%  CDAOtools.tex            ver: 1.00 %
%                                       %

%%IMPORTANT: do not delete the first line of this template
%%It must be present to enable the BMC Submission system to 
%%recognise this template!!

%%%%%%%%%%%%%%%%%%%%%%%%%%%%%%%%%%%%%%%%%
%%                                     %%
%%  LaTeX template for BioMed Central  %%
%%     journal article submissions     %%
%%                                     %%
%%         <14 August 2007>            %%
%%                                     %%
%%                                     %%
%% Uses:                               %%
%% cite.sty, url.sty, bmc_article.cls  %%
%% ifthen.sty. multicol.sty		   %%
%%				      	   %%
%%                                     %%
%%%%%%%%%%%%%%%%%%%%%%%%%%%%%%%%%%%%%%%%%


%%%%%%%%%%%%%%%%%%%%%%%%%%%%%%%%%%%%%%%%%%%%%%%%%%%%%%%%%%%%%%%%%%%%%
%%                                                                 %%	
%% For instructions on how to fill out this Tex template           %%
%% document please refer to Readme.pdf and the instructions for    %%
%% authors page on the biomed central website                      %%
%% http://www.biomedcentral.com/info/authors/                      %%
%%                                                                 %%
%% Please do not use \input{...} to include other tex files.       %%
%% Submit your LaTeX manuscript as one .tex document.              %%
%%                                                                 %%
%% All additional figures and files should be attached             %%
%% separately and not embedded in the \TeX\ document itself.       %%
%%                                                                 %%
%% BioMed Central currently use the MikTex distribution of         %%
%% TeX for Windows) of TeX and LaTeX.  This is available from      %%
%% http://www.miktex.org                                           %%
%%                                                                 %%
%%%%%%%%%%%%%%%%%%%%%%%%%%%%%%%%%%%%%%%%%%%%%%%%%%%%%%%%%%%%%%%%%%%%%


\NeedsTeXFormat{LaTeX2e}[1995/12/01]
\documentclass[10pt]{bmc_article}    



% Load packages
\usepackage{cite} % Make references as [1-4], not [1,2,3,4]
\usepackage{url}  % Formatting web addresses  
\usepackage{ifthen}  % Conditional 
\usepackage{multicol}   %Columns
\usepackage[utf8]{inputenc} %unicode support
%\usepackage[applemac]{inputenc} %applemac support if unicode package fails
%\usepackage[latin1]{inputenc} %UNIX support if unicode package fails
\urlstyle{rm}
 
 
%%%%%%%%%%%%%%%%%%%%%%%%%%%%%%%%%%%%%%%%%%%%%%%%%	
%%                                             %%
%%  If you wish to display your graphics for   %%
%%  your own use using includegraphic or       %%
%%  includegraphics, then comment out the      %%
%%  following two lines of code.               %%   
%%  NB: These line *must* be included when     %%
%%  submitting to BMC.                         %% 
%%  All figure files must be submitted as      %%
%%  separate graphics through the BMC          %%
%%  submission process, not included in the    %% 
%%  submitted article.                         %% 
%%                                             %%
%%%%%%%%%%%%%%%%%%%%%%%%%%%%%%%%%%%%%%%%%%%%%%%%%                     


\def\includegraphic{}
\def\includegraphics{}



\setlength{\topmargin}{0.0cm}
\setlength{\textheight}{21.5cm}
\setlength{\oddsidemargin}{0cm} 
\setlength{\textwidth}{16.5cm}
\setlength{\columnsep}{0.6cm}

\newboolean{publ}

%%%%%%%%%%%%%%%%%%%%%%%%%%%%%%%%%%%%%%%%%%%%%%%%%%
%%                                              %%
%% You may change the following style settings  %%
%% Should you wish to format your article       %%
%% in a publication style for printing out and  %%
%% sharing with colleagues, but ensure that     %%
%% before submitting to BMC that the style is   %%
%% returned to the Review style setting.        %%
%%                                              %%
%%%%%%%%%%%%%%%%%%%%%%%%%%%%%%%%%%%%%%%%%%%%%%%%%%
 

%Review style settings
\newenvironment{bmcformat}{\begin{raggedright}\baselineskip20pt\sloppy\setboolean{publ}{false}}{\end{raggedright}\baselineskip20pt\sloppy}

%Publication style settings
%\newenvironment{bmcformat}{\fussy\setboolean{publ}{true}}{\fussy}



% Begin ...
\begin{document}
\begin{bmcformat}


%%%%%%%%%%%%%%%%%%%%%%%%%%%%%%%%%%%%%%%%%%%%%%
%%                                          %%
%% Enter the title of your article here     %%
%%                                          %%
%%%%%%%%%%%%%%%%%%%%%%%%%%%%%%%%%%%%%%%%%%%%%%

\title{CDAO Store: A New Vision for Data Integration}
 
%%%%%%%%%%%%%%%%%%%%%%%%%%%%%%%%%%%%%%%%%%%%%%
%%                                          %%
%% Enter the authors here                   %%
%%                                          %%
%% Ensure \and is entered between all but   %%
%% the last two authors. This will be       %%
%% replaced by a comma in the final article %%
%%                                          %%
%% Ensure there are no trailing spaces at   %% 
%% the ends of the lines                    %%     	
%%                                          %%
%%%%%%%%%%%%%%%%%%%%%%%%%%%%%%%%%%%%%%%%%%%%%%


\author{Brandon Chisham\correspondingauthor%
       \email{Brandon Chisham\correspondingauthor - bchisham@cs.nmsu.edu}%
      \and
         Trung Le\correspondingauthor%
         \email{Trung Le\correspondingauthor - tle@cs.nmsu.edu}%
    \and
	Enrico Pontelli \correspondingauthor%
	\email{Enrico Pontelli \correspondingauthor -epontell@cs.nmsu.edu}%
    \and
	Tran Son \correspondingauthor%
	\email{Tran Son \correspondingauthor - tson@cs.nmsu.edu}%
       and 
         Ben Wright \correspondingauthor%
         \email{Ben Wright \correspondingauthor - bwright@cs.nmsu.edu}%
      }
      

%%%%%%%%%%%%%%%%%%%%%%%%%%%%%%%%%%%%%%%%%%%%%%
%%                                          %%
%% Enter the authors' addresses here        %%
%%                                          %%
%%%%%%%%%%%%%%%%%%%%%%%%%%%%%%%%%%%%%%%%%%%%%%

\address{%
    Department of Computer Science, New Mexico State University, Las Cruces, New Mexico, USA
}%

\maketitle

%%%%%%%%%%%%%%%%%%%%%%%%%%%%%%%%%%%%%%%%%%%%%%
%%                                          %%
%% The Abstract begins here                 %%
%%                                          %%
%% The Section headings here are those for  %%
%% a Research article submitted to a        %%
%% BMC-Series journal.                      %%  
%%                                          %%
%% If your article is not of this type,     %%
%% then refer to the Instructions for       %%
%% authors on http://www.biomedcentral.com  %%
%% and change the section headings          %%
%% accordingly.                             %%   
%%                                          %%
%%%%%%%%%%%%%%%%%%%%%%%%%%%%%%%%%%%%%%%%%%%%%%


\begin{abstract}
        % Do not use inserted blank lines (ie \\) until main body of text.
        %This should not exceet 350 words and should be structured into separate sections headed Background, Results, Conclusions.  Please do not use abbreviations or references in the abstract.
	\paragraph*{Background:} The Comparative Data Analysis Ontology
\footnote{\url{http://www.evolutionaryontology.org}} is an ontology developed,
as part of the
EvoInfo\footnote{\url{https://www.nescent.org/wg_evoinfo/Main_Page}} and
EvoIO\footnote{\url{http://evoio.org/wiki/Main_Page}} groups supported by
NESCent\footnote{\url{http://www.nescent.org/index.php}},  to provide semantics
to the descriptions of data and transformations commonly found in the domain of
phylogenetic inference. The core concepts of the ontology enables the
description of phylogenetic trees and associated character data matrices.

     

 
        \paragraph*{Results:} Text for this section of the abstract \ldots

        \paragraph*{Conclusions:} Text for this section of the abstract \ldots
\end{abstract}



\ifthenelse{\boolean{publ}}{\begin{multicols}{2}}{}




%%%%%%%%%%%%%%%%%%%%%%%%%%%%%%%%%%%%%%%%%%%%%%
%%                                          %%
%% The Main Body begins here                %%
%%                                          %%
%% The Section headings here are those for  %%
%% a Research article submitted to a        %%
%% BMC-Series journal.                      %%  
%%                                          %%
%% If your article is not of this type,     %%
%% then refer to the instructions for       %%
%% authors on:                              %%
%% http://www.biomedcentral.com/info/authors%%
%% and change the section headings          %%
%% accordingly.                             %% 
%%                                          %%
%% See the Results and Discussion section   %%
%% for details on how to create sub-sections%%
%%                                          %%
%% use \cite{...} to cite references        %%
%%  \cite{koon} and                         %%
%%  \cite{oreg,khar,zvai,xjon,schn,pond}    %%
%%  \nocite{smith,marg,hunn,advi,koha,mouse}%%
%%                                          %%
%%%%%%%%%%%%%%%%%%%%%%%%%%%%%%%%%%%%%%%%%%%%%%




%%%%%%%%%%%%%%%%
%% Background %%
%%
\section*{Background}
\subsection*{CDAO}

  CDAO, Comparative Data Analysis Ontology, provides a framework for describing phylogenies and their associated
  character state matrices. It was developed as part of the Evolutionary Informatics working group along with the
  NeXML file format, and the PhyloWS Webservice standard, forming what the group called the EvoIO stack. CDAO forms
  the base of this stack defining the semantics for the data represented as NeXML files, or otherwise supplied by
  services implementing this set of standards. 

  CDAO is defined in terms of an OWL-DL ontology. It provides a general framework
  for talking about the relationships between taxa, characters, states, their matrices, and associated 
  phylogenies. As a general framework it supplies general classes and relations between those classes, 
  it is intended that for practical work these will be extended to for example talk about more specific
  types of characters or states. (e.g. Beak length might be defined as a specialization of CDAO's \textit{Standard} 
  character type). 

\subsection*{NeXML}
  NeXML\footnote{\url{http://www.nexml.org}} is a file format for exchanging data containing character state data
  matrices and phylogenies. It is syntax is defined in terms of an XML schema, and the semantics of its elements
  are defined in terms of CDAO classes. Being defined in this way allows direct translation to CDAO class instances.
  This guarantee is also important to using it as a medium of exchange since it's semantics can be agreed upon by
  both the provider and recipient of a dataset.

\subsection*{PhlyoWS}
  PhyloWS is a standard for exposing phylogenetic data as a webservice, in such a way that particular data items,
  can be referenced by persistent HTTP URI's.

%%%%%%%%%%%%%%%%
%%Implementation %%
%%
\section*{Implementation}

CDAO-store builds on the EvoIO technology stack to provide a framework for supplying semantic services
for phylogenetic data services. The platform is open-source and is available on source-forge, at
\url{http://sourceforge.net/projects/cdaotools/}. It's divided into three main parts. A data importer/file translator,
a database and web interface, and a gui visualization tool. 

The file importer/translator is implemented in C++ and Python. In addition to its own set of parsers, 
the translator uses the NCL\footnote{\url{http://sourceforge.net/projects/ncl/}} library to read certain file formats.
After reading, it maps data from these files on to an object model that mirrors CDAO classes, and then either converts
to some specified format or to an RDF/XML serialization of the data. The import portion of this part of the system is
written in Python and uses the RDFlib\footnote{\url{http://www.rdflib.net/}} module to store the RDF serializations 
produced by the translator into a database making it available to query on the web or by using the visual tools.

The web and database portion of the application stores, and provides access to the data for the visual tools. This
portion of the application is primarily implemented as a set of scripts in a variety of languages. The web interface
is divided into two principal parts an HTML user interface, and a PhyloWS data provider. The HTML interface allows 
for online querying/exploration of datasets, while the PhyloWS interface supplies access to datasets for our visual
tools or other third party programs. The database portion of the application is implemented as an RDFlib store running
on a MySQL database. 

The visual tools are implemented as a Java JNLP application called CDAO-Explorer. It uses a variety of frameworks
to support its operation including Pellet\footnote{\url{http://pellet.owldl.com/}} and
  Prefuse\footnote{\url{http://prefuse.org/}}. 

CDAO-Explorer provides a tree and matrix search windows which allow one to search for and load particular datasets,
and visualizers for those data sets. It also allows one to make annotations about a dataset, or a general project space,
a set of data sets of interest. These annotations can be from CDAO, Dublin-Core, or from a user-supplied source of
annotation types.



 
%%%%%%%%%%%%%%%%%%%%%%%%%%%%
%% Results and Discussion %%
%%
\section*{Results}
 % What should be described here is the functionality of the software together with data on  how its performance and functionality compare with and improve on functionally similar existing software.
  \subsection*{Web-Tools}
  The web tools provide a variety of querying and data access features for both human and programmatic access to
  data. It allows one to retrieve data sets by author name, tree identifier, taxon, algorithm, or method. It also
  supports computing minimum spanning clades, the nearest common ancestor of a set of taxa, or listing trees of 
  various sizes. 
  
  \subsection*{CDAO-Explorer}
  CDAO-Explorer has achieved a basic level of functionality. It provides searching and visualization of both
  tree and matrix data. We believe this integration between tree and matrix viewing to be a novel feature. The
  support for supplementing data with additional user-defined annotations. Additionally, it allows edges to be
  treated as first-class data entities, which can have annotations such as length bootstrap support attached.


%Things to compare functionality with: Nexplorer and PhiloWidget  (about the same functionality with trees, but they dont display matrices and viewing edges as data)

\section*{Discussion}
 % Intended use of the software and the benefits that are envisioned together, if possible, with an outline for the planned future development of new features.

  With this basic level of functionality in place, we envision extending CDAO tools to include support for describing
  workflows in cooperation with the MIAPA\footnote{Minimim Information About a Phylogenetic Analysis} effort.
  We also plan to add additional query features to the web-interface including the ability to process user supplied
  SPARQL Queries.
  
  %Include conversation on MIAPA and OBI?

%%%%%%%%%%%%%%%%%%%%%%
\section*{Conclusions}
  Text for this section \ldots

	
  
%%%%%%%%%%%%%%%%%%
\section*{Availability and Requirements}
   Project name: CDAO Tools
   Project home page: http://www.cs.nmsu.edu/~cdaostore/
  Operating system(s): Linux, Mac, Unix
  Programming language: Bash, C++, Java, Perl, PHP, Python
  Other requirements:
  License: GPL
  Any restriction to use by non-academics:

    
%%%%%%%%%%%%%%%%%%%%%%%%%%%%%%%%
\section*{Authors contributions}
    Text for this section \ldots

    

%%%%%%%%%%%%%%%%%%%%%%%%%%%
\section*{Acknowledgements}
  \ifthenelse{\boolean{publ}}{\small}{}
  Text for this section \ldots


 
%%%%%%%%%%%%%%%%%%%%%%%%%%%%%%%%%%%%%%%%%%%%%%%%%%%%%%%%%%%%%
%%                  The Bibliography                       %%
%%                                                         %%              
%%  Bmc_article.bst  will be used to                       %%
%%  create a .BBL file for submission, which includes      %%
%%  XML structured for BMC.                                %%
%%  After submission of the .TEX file,                     %%
%%  you will be prompted to submit your .BBL file.         %%
%%                                                         %%
%%                                                         %%
%%  Note that the displayed Bibliography will not          %% 
%%  necessarily be rendered by Latex exactly as specified  %%
%%  in the online Instructions for Authors.                %% 
%%                                                         %%
%%%%%%%%%%%%%%%%%%%%%%%%%%%%%%%%%%%%%%%%%%%%%%%%%%%%%%%%%%%%%


{\ifthenelse{\boolean{publ}}{\footnotesize}{\small}
 \bibliographystyle{bmc_article}  % Style BST file
  \bibliography{bmc_article} }     % Bibliography file (usually '*.bib' ) 

%%%%%%%%%%%

\ifthenelse{\boolean{publ}}{\end{multicols}}{}

%%%%%%%%%%%%%%%%%%%%%%%%%%%%%%%%%%%
%%                               %%
%% Figures                       %%
%%                               %%
%% NB: this is for captions and  %%
%% Titles. All graphics must be  %%
%% submitted separately and NOT  %%
%% included in the Tex document  %%
%%                               %%
%%%%%%%%%%%%%%%%%%%%%%%%%%%%%%%%%%%

%%
%% Do not use \listoffigures as most will included as separate files

\section*{Figures}
  \subsection*{Figure 1 - Sample figure title}
      A short description of the figure content
      should go here.

  \subsection*{Figure 2 - Sample figure title}
      Figure legend text.



%%%%%%%%%%%%%%%%%%%%%%%%%%%%%%%%%%%
%%                               %%
%% Tables                        %%
%%                               %%
%%%%%%%%%%%%%%%%%%%%%%%%%%%%%%%%%%%

%% Use of \listoftables is discouraged.
%%
\section*{Tables}
  \subsection*{Table 1 - Sample table title}
    Here is an example of a \emph{small} table in \LaTeX\ using  
    \verb|\tabular{...}|. This is where the description of the table 
    should go. \par \mbox{}
    \par
    \mbox{
      \begin{tabular}{|c|c|c|}
        \hline \multicolumn{3}{|c|}{My Table}\\ \hline
        A1 & B2  & C3 \\ \hline
        A2 & ... & .. \\ \hline
        A3 & ..  & .  \\ \hline
      \end{tabular}
      }
  \subsection*{Table 2 - Sample table title}
    Large tables are attached as separate files but should
    still be described here.



%%%%%%%%%%%%%%%%%%%%%%%%%%%%%%%%%%%
%%                               %%
%% Additional Files              %%
%%                               %%
%%%%%%%%%%%%%%%%%%%%%%%%%%%%%%%%%%%

\section*{Additional Files}
  \subsection*{Additional file 1 --- Sample additional file title}
    Additional file descriptions text (including details of how to
    view the file, if it is in a non-standard format or the file extension).  This might
    refer to a multi-page table or a figure.

  \subsection*{Additional file 2 --- Sample additional file title}
    Additional file descriptions text.


\end{bmcformat}
\end{document}







